\documentclass[12pt, a4paper]{article}

\usepackage[ngerman]{babel} 
\usepackage[T1]{fontenc}
\usepackage{amsfonts} 
\usepackage{setspace}
\usepackage{amsmath}
\usepackage{amssymb}
\usepackage{titling}


\newcommand*{\qed}{\null\nobreak\hfill\ensuremath{\square}}
\newcommand*{\puffer}{\text{ }\text{ }\text{ }\text{ }}
\newcommand*{\gedanke}{\textbf{-- }}
\newcommand*{\gap}{\text{ }}
\newcommand*{\vor}{\textbf{Vor.:} \gap}
\newcommand*{\beh}{\textbf{Beh.:} \gap}
\newcommand*{\bew}{\textbf{Bew.:} \gap}


\pagestyle{plain}
\allowdisplaybreaks

\setlength{\droptitle}{-14em}
\setlength{\jot}{12pt}

\title{Mathematik für die Informatik C\\Hausaufgabenserie 1}
\author{Henri Heyden, Nike Pulow \\ \small stu240825, stu239549}
\date{}


\begin{document}
\maketitle

\doublespacing
\subsection*{A1}
    \vor
        \(c \in [0, +\infty[\), \gap \(f : \mathbb R \rightarrow \mathbb R, x \mapsto c\cdot (x^2-3)\cdot e^{4x+1}\) \\
        \(I_1 := ]-\infty, -2]; \gap I_2 := [-2, \frac{3}{2}]; \gap I_3 := [\frac{3}{2}, +\infty[\) \\
    \beh \\
        Fall 1. \(c \in ]0, +\infty[\): \\ \(f\) ist streng monoton steigend in \(I_1, I_3\) und streng monoton fallend in \(I_2\) \\
        Somit gelte \(LMAX(f) = \{-2\}, \gap LMIN(f) = \{\frac{3}{2}\}\) \\
        Fall 2. \(c = 0\): \\ \(f\) ist konstant in jedem Punkt, \(LMAX(f) = LMIN(f) = \emptyset\). \\
    \bew Wir wissen \(f\) ist differenzierbar in ihrer Domain durch den Kombinationssatz der Differenzierbarkeit. Es folgt durch bekannte Regeln die Ableitung: \(f'(x) = 2c \cdot e^{4x+1} \cdot (2x^2+x-6)\).\\
    Für Fall 2 ist der Beweis trivial, denn es gilt \(f'(x) = 0\) für jedes \(x \in \mathbb R\) somit folgt die Behauptung nach dem Monotoniekriterium. \\
    Nun betrachten wir Fall 1. Da \(c > 0\) wissen wir, dass \(c\) für die Vorzeichen der Ableitung zu vernachlässigen ist. Des Weiteren ist bekannt, dass \(e^{4x+1}\) auch für \(x \in \mathbb R\) keine Nullstellen annimmt und immer positiv ist. \\
    Definiere \(\phi : \mathbb R \rightarrow \mathbb R, x \mapsto 2x^2+x-6\). Nach vorheriger Überlegung wissen wir \(f'(x)\) ist genau dann \(0\), wenn \(\phi(x) = 0\) gilt, sowie sind die Vorzeichen für alle \(x\in \mathbb R\) gleich. \\
    Für \(\phi\) ergeben sich die Nullstellen \(\phi^{\leftarrow}(0) = \{\frac{3}{2}, -2\}\). \\
    Es gilt: \(\phi(-3) = 9 > \phi(-2) = 0 > \phi(0) = -6 < \phi(\frac{3}{2}) = 0 < \phi(2) = 4\).\\ 
    Die Behauptung folgt aus dem Monotoniekriterium. \qed
\subsection*{A2}
\textbf{Vor.:} $f : \mathbb{R}_{\neq 0} \rightarrow \mathbb{R}$ ist differenzierbar. Es gilt $f'(x) > 0$ für alle $x \in \mathbb{R}_{\neq 0}$.\\
$f$ ist stetig fortsetzbar in $0$.\\
\textbf{Beh.:} $f$ ist streng monoton steigend. \\
\textbf{Bew.:} Nach Voraussetzung ist $f$ differenzierbar, also stetig. Es gilt nach Voraussetzung $f'(x) > 0$ 
für alle $x \in \mathbb{R}_{\neq 0}$, folglich ist $f$ wegen des Monotoniekriteriums streng monoton steigend in $]- \infty , 0 [$ 
und $]0, + \infty$.
Wegen der stetigen Fortsetzbarkeit von $f$ nach $0$ wissen wir auch, dass
\begin{flalign*}
    \mathbb{R}_{\neq 0} \cup \{0\} \rightarrow \mathbb{R}, x \mapsto \begin{cases}
        f(x) &\text{$x \in \mathbb{R}_{\neq 0}$}\\
        y^* &\text{$x = 0$}
    \end{cases}
\end{flalign*}
eine stetige Funktion ist. Da also $f$ in jedem Punkt stetig ist, ist $f$ streng monoton steigend.\\
\qed
\end{document}
