\documentclass[12pt, a4paper]{article}

\usepackage[ngerman]{babel} 
\usepackage[T1]{fontenc}
\usepackage{amsfonts} 
\usepackage{setspace}
\usepackage{amsmath}
\usepackage{amssymb}
\usepackage{titling}
\usepackage{hyperref}


\newcommand*{\qed}{\null\nobreak\hfill\ensuremath{\square}}
\newcommand*{\puffer}{\text{ }\text{ }\text{ }\text{ }}
\newcommand*{\gedanke}{\textbf{-- }}
\newcommand*{\gap}{\text{ }}
\newcommand*{\setDef}{\gap|\gap}
\newcommand*{\vor}{\textbf{Vor.:} \gap}
\newcommand*{\beh}{\textbf{Beh.:} \gap}
\newcommand*{\bew}{\textbf{Bew.:} \gap}
% Hab länger gebraucht um zu realisieren, dass das ne gute Idee wäre
\newcommand*{\R}{\mathbb R}


\pagestyle{plain}
\allowdisplaybreaks

\setlength{\droptitle}{-14em}
\setlength{\jot}{12pt}

\title{Mathematik für die Informatik C\\Hausaufgabenserie 3}
\author{Henri Heyden, Nike Pulow \\ \small stu240825, stu239549}
\date{}


\begin{document}
\maketitle

\doublespacing

\subsection*{A1}
\vor \(a,b,t \in ]0,+\infty[, \gap a < b, \gap f, g' \in \R ^ {]0,+\infty[}, \gap f(x) = \ln x, \gap g'(x) = x^{-t}\). \\
\beh \(\int_a^b \frac{ln(x)}{x^t} = \left[ \frac{x^{1-t}}{1-t} \cdot \left(\ln(x) - \frac{1}{1-t}\right) \right]_a^b\). \\
\bew Es gilt: \(\int_a^b \frac{ln(x)}{x^t} = \int_{a}^{b} (f \cdot g') (x)\). Dann können wir mittels partieller Integration schreiben:
\begin{flalign*}
    \int_{a}^{b} (f \cdot g') (x) & = \left[(f \cdot g)(x)\right]_a^b - \int_a^b (f'\cdot g)(x) & \text{| Einsetzen und ausrechnen} \\
    & = \left[\ln(x) \cdot \frac{x^{1-t}}{1-t}\right]_a^b - \int_{a}^{b} \frac{1}{x} \cdot \frac{x^{1-t}}{1-t} & \text{| Rausziehen und Potenzgesetze} \\
    & = \left[\ln(x) \cdot \frac{x^{1-t}}{1-t}\right]_a^b - \frac{1}{1-t} \cdot \int_a^b x^{-t} & \text{| Ausrechnen und Reinziehen} \\
    & = \left[\ln(x) \cdot \frac{x^{1-t}}{1-t}\right]_a^b -\left[ \frac{1}{1-t} \cdot \frac{x^{1-t}}{1-t}\right]_a^b & \text{| Zusammenziehen und Ausklammern} \\
    & = \left[ \frac{x^{1-t}}{1-t} \cdot \left(\ln(x) - \frac{1}{1-t}\right) \right]_a^b
\end{flalign*} \qed

\end{document}