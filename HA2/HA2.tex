\documentclass[12pt, a4paper]{article}

\usepackage[ngerman]{babel} 
\usepackage[T1]{fontenc}
\usepackage{amsfonts} 
\usepackage{setspace}
\usepackage{amsmath}
\usepackage{amssymb}
\usepackage{titling}
\usepackage{hyperref}


\newcommand*{\qed}{\null\nobreak\hfill\ensuremath{\square}}
\newcommand*{\puffer}{\text{ }\text{ }\text{ }\text{ }}
\newcommand*{\gedanke}{\textbf{-- }}
\newcommand*{\gap}{\text{ }}
\newcommand*{\setDef}{\gap|\gap}
\newcommand*{\vor}{\textbf{Vor.:} \gap}
\newcommand*{\beh}{\textbf{Beh.:} \gap}
\newcommand*{\bew}{\textbf{Bew.:} \gap}
% Hab länger gebraucht um zu realisieren, dass das ne gute Idee wäre
\newcommand*{\R}{\mathbb R}


\pagestyle{plain}
\allowdisplaybreaks

\setlength{\droptitle}{-14em}
\setlength{\jot}{12pt}

\title{Mathematik für die Informatik C\\Hausaufgabenserie 2}
\author{Henri Heyden, Nike Pulow \\ \small stu240825, stu239549}
\date{}


\begin{document}
\maketitle

\doublespacing

\subsection*{A1}
\vor \(f: [-1, 1] \rightarrow \R, x \mapsto \begin{cases}
    1 & x = 0 \\
    0 & \text{sonst}
\end{cases}\) \\
\beh \(f\) ist integrierbar und es gilt: \(\int f = 0\). \\
\bew \\Wir werden zeigen: \[\forall (x^{(n)}, \xi^{(n)})_n \in \mathcal S(\text{PS} (a,b)), \lim_n \mu(x^{(n)}) = 0 : \lim_{n} R(f, x^{(n)}, \xi^{(n)}) = 0\]
Nehme also an, dass wir zwei beliebige Folgen \((x^{(n)}, \xi^{(n)})_n \in \mathcal S(\text{PS} (a,b))\) haben für die gilt: \(\lim_n \mu(x^{(n)}) = 0\). \\
Für jedes \(\xi\) (Hier alle Tupel, also alle Folgekomponenten aller der eben referierten Folgen) gibt es zwei Fälle: \\
\textbf{Fall 1}: \(1 \not\in f^\rightarrow(\xi)\). \\
Dann gilt: \(\forall \phi \in \xi: \phi \ne 0\). Hieraus lässt sich schließen \(f^\rightarrow(\xi) = \{0\}\) \\
Dann sieht man leicht, dass \(\lim_{n} R(f, x^{(n)}, \xi^{(n)}) = 0\), gilt aufgrund der Definition der Riemann-Summe. \\
\textbf{Fall 2}: \(1 \in f^\rightarrow(\xi)\). \\
Dann gilt: \(0 \in \xi\) und \(f^\leftarrow (1) = 0\). Es existiert also genau eine \(\phi \in \xi\), sodass \(f(\phi) = 1\) gilt mit \(\phi = 0\). \\
Definiere folgende Mengen:\\
\(\Gamma_n := \{\phi \in \xi \setDef f(\phi) = 1\}, \gap \Delta_n := \{\phi \in \xi \setDef f(\phi) = 0\}\). \\
Nach vorheriger Überlegung gilt \(|\Gamma| = 1\) und man sieht leicht \(|\Delta| = n - 1\). \\
Jetzt können wir zeigen, dass \(\lim_{n} R(f, x^{(n)}, \xi^{(n)}) = 0\) gilt\footnote[1]{Im 2. Schritt meinen wir damit, dass wenn die Komponenten der Tupel so getauscht werden, dass der Wert für \(\xi_1 = \phi = 0\) gilt, dass dann sich das Integral nicht ändert.}:
\begin{flalign*}
    \lim_{n} R(f, x^{(n)}, \xi^{(n)}) & = \lim_{n} \sum_{j = 1}^{n} l(x^{(n)}, j) \cdot f(\xi^{(n)}_j) \puffer \puffer \text{| Reihenfolge der Summe ist egal, \(|\Gamma|, |\Delta| \)} & \\
    & = \lim_{n} \left( \sum_{j = 1}^{1} l(x^{(n)}, j) \cdot f(\phi) + \sum_{j = 2}^{n} l(x^{(n)}, j) \cdot f(\xi^{(n)}_j) \right) & \\
    & = \lim_{n} \left( \sum_{j = 1}^{1} l(x^{(n)}, j) \cdot f(0) + \sum_{j = 2}^{n} l(x^{(n)}, j) \cdot f(1) \right) \puffer \text{| Auswerten} \\
    & = \lim_{n} \left( \sum_{j = 1}^{1} l(x^{(n)}, j) \cdot 1) + \sum_{j = 2}^{n} l(x^{(n)}, j) \cdot 0 \right) & \\
    & = \lim_{n} \sum_{j = 1}^{1} l(x^{(n)}, j)) & \\
    & = \lim_{n} l(x^{(n)}, j) \puffer \puffer \puffer \puffer \puffer \puffer \text{|Def. \(\mu\), \(\mu \rightarrow 0\)} \\
    & = 0
\end{flalign*}
Nach Konvergenzkriterium ist somit \(f\) integrierbar mit Integral \(\int f = 0\)  \qed \pagebreak
\subsection*{A2}
\vor \(n \in \mathbb N_1\), definiere \(\delta^{(n)} := \frac{b-a}{n}\) sowie \((x^{(n)}, \xi ^ {(n)}) \in \text{PS}(a,b)\) mit:
\begin{flalign*}
    & \puffer x_j ^{(n)} := a + j\delta ^ {(n)} & j \in [n]_0 \\
    & \puffer \xi_j ^{(n)} := a + (j - 1) \delta ^ {(n)} & j \in [n]_1
\end{flalign*}
Definiere \(f: [a,b] \rightarrow \R, x \mapsto e^x\).
\subsubsection*{1)}
\beh Es gilt \(R(f, x^{(n)}, \xi^{(n)}) = \delta^{(n)}\frac{e^b-e^a}{e^{\delta^{(n)}}-1}\).\\
\bew
Zunächst wenden wir die Definition der Riemann-Summe an und formen dann so um, dass die Geometrische Summenformel anwendbar ist:
\begin{flalign*}
    R(f, x^{(n)}, \xi^{(n)}) &= \sum_{j=1}^{n}(x_{j}-x_{j-1})f(\xi_j) & \text{Def. \(x_j, \xi_j\)}\\
    \begin{split}
        =\sum_{j=1}^{n}((a+j\frac{b-a}{n})-(a+(j-1)\frac{b-a}{n})) \\
        f(a+(j-1)\frac{b-a}{n})\\
    \end{split}\\
    &=\sum_{j=1}^{n}\frac{b-a}{n}f(a+(j-1)\frac{b-a}{n}) & \text{Def. \(\delta^{(n)}\), Konstante}\\
    &=\delta^{(n)}\sum_{j=1}^{n}f(a+(j-1)\frac{b-a}{n}) & \text{\(f\) anwenden, Def. \(\delta^{(n)}\)}\\
    &=\delta^{(n)}\sum_{j=1}^{n}e^a{(e^{\delta^{(n)}})}^{j-1} & \text{\(e^a\) Konstante}\\
    &=\delta^{(n)} e^a \sum_{j=1}^{n} {(e^{\delta^{(n)}})}^{j-1} & q:=e^{\delta^{(n)}}\\
    &=\delta^{(n)} e^a \sum_{j=1}^{n}q^{j-1} & \text{Geometrische Summenformel}\\
    &=\delta^{(n)} e^a \frac{q^n-1}{q-1} & \text{Def. \(q\)}\\
    &=\delta^{(n)} e^a \frac{{(e^{\delta^{(n)}})}^n-1}{e^{\delta^{(n)}}-1} & \text{Def.} \delta^{(n)}\\
    &=\delta^{(n)} e^a \frac{{(e^{\frac{b-a}{n}})}^n-1}{e^{\delta^{(n)}}-1} & \text{Potenzregeln}\\
    &=\delta^{(n)} e^a \frac{e^{b-a}-1}{e^{\delta^{(n)}}-1} & \text{Bruchregeln}\\
    &=\delta^{(n)} \frac{e^a}{1} \frac{e^{b-a}-1}{e^{\delta^{(n)}}-1} & \text{Ausmultiplizieren}\\
    &=\delta^{(n)} \frac{e^a e^{b-a}-e^a}{e^{\delta^{(n)}}-1} & \\
    &=\delta^{(n)} \frac{e^{b-a}}{e^{\delta^{(n)}}-1} & \text{Potenzregeln}\\
    &=\delta^{(n)} \frac{e^b - e^a}{e^{\delta^{(n)}}-1} &
\end{flalign*}
Damit ist gezeigt, was zu zeigen war.\\
\qed
\pagebreak
\subsubsection*{2)}
\beh \(\lim_{n} R(f, x^{(n)}, \xi^{(n)}) = e^b - e^a\) \\
\bew Es gilt nach vorheriger Aufgabe: \(\lim_{n} R(f, x^{(n)}, \xi^{(n)}) = \lim_{n} \left( \delta ^ {(n)} \cdot \frac{e^b - e^a}{{e^\delta}^{(n)} - 1} \right)\). \\
Betrachte folgende Umformung\footnote[2]{Wenn wir die Regel von L'Hôpital anwenden, dann betrachten wir nicht mehr eine Folge, sondern ALLE Funktionswertfolgen. \(n\) wird zur Folgenkomponente einer Folge mit Limes \(+\infty\). Jedoch da der Funktionslimes dann für alle Funktionswertfolgen gilt, gilt er auch für die Folge \((id_\mathbb{N}(k))_k\). Da wir alle Funktionswertfolgen betrachten, nimmt \(n\) also auch diese Werte an, weswegen wir, wenn wir für alle Funktionswertfolgen den Funktionslimes zeigen, damit dann auch den Limes der originalen Folge zeigen, da diese eine Teilfolge der Funktionswertfolgen ist. Eine Diskussion über dieses Thema lässt sich \href{https://www.matheplanet.com/default3.html?call=viewtopic.php?topic=135777}{hier} nachlesen. \\ Alternativ lässt sich auch der Satz von Stolz anwenden.}:
\begin{flalign*}
    & \lim_{n} \left( \delta ^ {(n)} \cdot \frac{e^b - e^a}{{e^\delta}^{(n)} - 1} \right) & \\
    & = (e^b - e^a) \cdot  \lim_{n} \left(\frac{\delta ^ {(n)}}{{e^\delta}^{(n)} - 1} \right) & \text{| Einsetzen} \\
    & = (e^b - e^a) \cdot \lim_{n} \left( \frac{\frac{b-a}{n}}{e^\frac{b-a}{n} - 1} \right) & \text{| Stetigkeit, L'Hôpital für \(id_{\mathbb N}\)} \\
    & = (e^b - e^a) \cdot \lim_{n \rightarrow +\infty} \left( \frac{\frac{a-b}{n^2}}{\frac{(a-b) \cdot e^{\frac{b-a}{n}}}{n^2}} \right) & \text{| Zwei mal kürzen} \\
    & = (e^b - e^a) \cdot \lim_{n \rightarrow +\infty} \left( \frac{1}{e^{\frac{b-a}{n}}} \right) & \text{| Stetigkeit, \(\frac{b-a}{n} \rightarrow 0\)} \\
    & = (e^b - e^a) \cdot \lim_{n \rightarrow +\infty} \left( \frac{1}{e^0} \right) & \\
    & = (e^b - e^a)
\end{flalign*}
Was zu zeigen war. \qed
\end{document}
