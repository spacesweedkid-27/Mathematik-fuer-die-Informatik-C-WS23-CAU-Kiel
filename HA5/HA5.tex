\documentclass[12pt, a4paper]{article}

\usepackage[ngerman]{babel} 
\usepackage[T1]{fontenc}
\usepackage{amsfonts} 
\usepackage{setspace}
\usepackage{amsmath}
\usepackage{amssymb}
\usepackage{titling}
\usepackage{hyperref}


\newcommand*{\qed}{\null\nobreak\hfill\ensuremath{\square}}
\newcommand*{\puffer}{\text{ }\text{ }\text{ }\text{ }}
\newcommand*{\gedanke}{\textbf{-- }}
\newcommand*{\gap}{\text{ }}
\newcommand*{\setDef}{\gap|\gap}
\newcommand*{\vor}{\textbf{Vor.:} \gap}
\newcommand*{\beh}{\textbf{Beh.:} \gap}
\newcommand*{\bew}{\textbf{Bew.:} \gap}
% Hab länger gebraucht um zu realisieren, dass das ne gute Idee wäre
\newcommand*{\R}{\mathbb R}


\pagestyle{plain}
\allowdisplaybreaks

\setlength{\droptitle}{-14em}
\setlength{\jot}{12pt}
\setlength{\hoffset}{-2cm}


\title{Mathematik für die Informatik C\\Hausaufgabenserie 5}
\author{Henri Heyden, Nike Pulow \\ \small stu240825, stu239549}
\date{}


\begin{document}
\maketitle

\doublespacing

\subsection*{A1}
\vor Definiere \(f : \mathbb{R} \rightarrow \mathbb{R}, x \mapsto
\begin{cases}
    e^x & x < 0\\
    1-x & \text{sonst}\\
\end{cases}\)  und \(S=T=\mathbb{R}\).\\
\beh \(f\) ist stetig.\\
\bew Definiere \(A:=]- \infty, 0 [\) und \(B := [0,+ \infty[\) und bemerke \(S=A \cup B = \mathbb{R}\). 
Nun zeigen wir, dass \(f |_A\) und \(f |_B\) stetig sind:\\
(1) Für \(f |_A\) gilt: \\
\(f(x) = e^x\) wegen der Funktionsdefinition und \(x<0\). Die Exponentialfunktion ist stetig auf 
ganz \(\mathbb{R}\), das ist trivial, also insbesondere auch stetig auf \(A\).\\
(2) Für \(f|_B\) gilt: \\
\(f(x) = 1-x\) wegen der Funktionsdefinition und \(x \geq 0\). Offenbar ist \(f\) auf \(B\) also 
eine lineare Funktion und deren Stetigkeit trivial.\\
Da \(f\) sowohl auf \(A\), als auch auf \(B\) stetig ist, also auf allen Teilen von \(S\) 
stetig ist, ist folglich auch \(f\) stetig. \qed

\subsection*{A2}
\vor \(A : = [0,1]\), \(\mathbb{R}\) ist ein metrischer Raum.\\
\beh Nicht jede abgeschlossene Menge, die Teilmenge einer kompakten Mengen ist, ist auch kompakt.\\
\bew \(A\) ist offenbar abgeschlossen, da ihr Komplement \(]-\infty,0[\cup]1,+\infty[\) offen ist, 
und folglich kompakt, da \(A\) abgeschlossen und beschränkt ist. Definiere nun \(B:=]0,1]\). Es gilt 
offenbar \(B \subset A\). Wie leicht erkennbar ist, ist \(B\) nicht kompakt, da das linksseitige 
Komplement \(]-\infty,0]\) nicht offen ist und es sich bei \(B\) somit nicht um ein 
abgeschlossenes Intervall handelt. \qed

\subsection*{A3}
\vor Definiere \(\)
\end{document}