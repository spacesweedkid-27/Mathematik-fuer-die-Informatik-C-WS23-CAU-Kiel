\documentclass[a4paper, 12pt]{article}

\usepackage[ngerman]{babel} 
\usepackage[T1]{fontenc}
\usepackage{amsfonts} 
\usepackage{setspace}
\usepackage{amsmath}
\usepackage{amssymb}
\usepackage{titling}
\usepackage{hyperref}
\usepackage{csquotes} % for \textquote{}
% REMINDER: USE IEEEeqnarray* FOR ALINGMENTS%
\usepackage{IEEEtrantools}
\usepackage{stix}

\newcommand*{\qed}{\null\nobreak\hfill\ensuremath{\square}}
\newcommand*{\puffer}{\text{ }\text{ }\text{ }\text{ }}
\newcommand*{\gedanke}{\textbf{-- }}
\newcommand*{\gap}{\text{ }}
\newcommand*{\setDef}{\gap|\gap}
\newcommand*{\vor}{\textbf{Voraussetzung} \gap}
\newcommand*{\beh}{\textbf{Behauptung} \gap}
\newcommand*{\bew}{\textbf{Beweis} \gap}
% Hab länger gebraucht um zu realisieren, dass das ne gute Idee wäre
\newcommand*{\R}{\mathbb R}
\newcommand*{\grad}{\text{grad}}
\newcommand*{\J}{\textbf{J}}
\newcommand*{\He}{\text{H}}
\newcommand*{\LMIN}{\text{\scshape Lmin}}
\newcommand*{\LMAX}{\text{\scshape Lmax}}

% ¯\_(ツ)_/¯
\usepackage{tikz}
\newcommand{\shrug}[1][]{%
\begin{tikzpicture}[baseline,x=0.8\ht\strutbox,y=0.8\ht\strutbox,line width=0.125ex,#1]
\def\arm{(-2.5,0.95) to (-2,0.95) (-1.9,1) to (-1.5,0) (-1.35,0) to (-0.8,0)};
\draw \arm;
\draw[xscale=-1] \arm;
\def\headpart{(0.6,0) arc[start angle=-40, end angle=40,x radius=0.6,y radius=0.8]};
\draw \headpart;
\draw[xscale=-1] \headpart;
\def\eye{(-0.075,0.15) .. controls (0.02,0) .. (0.075,-0.15)};
\draw[shift={(-0.3,0.8)}] \eye;
\draw[shift={(0,0.85)}] \eye;
% draw mouth
\draw (-0.1,0.2) to [out=15,in=-100] (0.4,0.95); 
\end{tikzpicture}}


\pagestyle{plain}
\allowdisplaybreaks

\setlength{\droptitle}{-14em}
\setlength{\jot}{12pt}

\title{\scshape Mathematik für die Informatik C\\Hausaufgabenserie 9}
\author{\scshape Henri Heyden, Nike Pulow \\ \small stu240825, stu239549}
\date{}

\renewcommand{\baselinestretch}{1.5} % singlehalfspacing

\begin{document}
\maketitle
\subsubsection*{\scshape A1}
\vor \\
Sei \(\xi,\eta,a,b,c \in \R\) mit \(\xi \cdot \eta > 0\)\\
und \(f: \R^2 \rightarrow \R, (x,y) \mapsto \xi(x-a)^2 + \eta(y-b)^2 + c\). \\
\beh\\
\(\LMAX(f) = \{(a,b)\}\), wenn \(\xi < 0\) und \(\LMIN(f) = \{(a,b)\}\), wenn \(\xi > 0\) \\
\bew \\
Nach dem Kombsatz. ist \(f\) eine \(\mathcal C^2\)-Funktion. \\
Zunächst forme \(f\) um. Es gilt:

% this space is important
{
    \setstretch{1}
    \begin{IEEEeqnarray*}{rCl"l}
        f(x,y) & = & \xi(x-a)^2 + \eta(y-b)^2 + c & \text{| Binom. Formeln} \\
        & = & \xi(x^2 -2xa +a^2) + \eta(y^2-2yb+b^2) + c & \text{| Klammern auflösen} \\
        & = & \xi x^2 -\xi 2xa +\xi a^2 + \eta y^2-\eta 2yb + \eta b^2 + c
    \end{IEEEeqnarray*}
    Dann gilt:
    \begin{IEEEeqnarray*}{"lClCl}
        \partial_1f(x) & = & 2\xi x - 2\xi a & = & 2\xi (x-a) \\
        \partial_2f(y) & = & 2\eta y - 2\eta b & = & 2\eta (y-b)
    \end{IEEEeqnarray*}
    Des Weiteren gilt:
    \begin{IEEEeqnarray*}{"lCl}
        \partial_1\partial_1f(x) & = & 2\xi \\
        \partial_2\partial_1f(y) & = & 0 \\
        \partial_1\partial_2f(x) & = & 0 \\
        \partial_2\partial_2f(y) & = & 2\eta
    \end{IEEEeqnarray*}
    Somit gilt: \(\He_f = \begin{bmatrix}
        2\xi & 0 \\ 0 & 2\eta
    \end{bmatrix}\) \\
    Außerdem gilt:
    \begin{IEEEeqnarray*}{"rClCrCl}
        \partial_1f(x) & = & 0 & \Longleftrightarrow & x & = & a  \\
        \partial_2f(y) & = & 0 & \Longleftrightarrow & y & = & b
    \end{IEEEeqnarray*}
}
Dann gilt also \(\partial_1f(a) = \partial_2f(b) = 0\) somit ist \((a,b)\) kritische Stelle. \\\\
\vspace*{-0.5cm}
Es gilt:

{   
    \setstretch{1}
    \begin{IEEEeqnarray*}{"rCl}
        \He_f(a,b) & = & \begin{bmatrix}
            2\xi & 0 \\ 0 & 2\eta
        \end{bmatrix}
    \end{IEEEeqnarray*}
}
und somit \(\det(\He_f(a,b)) = 2\xi \cdot 2\eta - 0\) \\
Da \(\xi \cdot \eta > 0\) gilt \(\det(\He_f(a,b)) > 0\), also ist \(\He_f(a,b)\) definit. \\
Somit gilt \((a,b) \in \LMAX(f)\), wenn \(\xi < 0\) \\
und \((a,b) \in \LMIN\), wenn \(\xi > 0\). \\
Die Behauptung folgt.
\qed
\end{document}