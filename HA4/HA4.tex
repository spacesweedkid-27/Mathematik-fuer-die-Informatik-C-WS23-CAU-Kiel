\documentclass[12pt, a4paper]{article}

\usepackage[ngerman]{babel} 
\usepackage[T1]{fontenc}
\usepackage{amsfonts} 
\usepackage{setspace}
\usepackage{amsmath}
\usepackage{amssymb}
\usepackage{titling}
\usepackage{hyperref}


\newcommand*{\qed}{\null\nobreak\hfill\ensuremath{\square}}
\newcommand*{\puffer}{\text{ }\text{ }\text{ }\text{ }}
\newcommand*{\gedanke}{\textbf{-- }}
\newcommand*{\gap}{\text{ }}
\newcommand*{\setDef}{\gap|\gap}
\newcommand*{\vor}{\textbf{Vor.:} \gap}
\newcommand*{\beh}{\textbf{Beh.:} \gap}
\newcommand*{\bew}{\textbf{Bew.:} \gap}
% Hab länger gebraucht um zu realisieren, dass das ne gute Idee wäre
\newcommand*{\R}{\mathbb R}


\pagestyle{plain}
\allowdisplaybreaks

\setlength{\droptitle}{-14em}
\setlength{\jot}{12pt}
\setlength{\hoffset}{-2cm}


\title{Mathematik für die Informatik C\\Hausaufgabenserie 4}
\author{Henri Heyden, Nike Pulow \\ \small stu240825, stu239549}
\date{}


\begin{document}
\maketitle

\doublespacing

\subsection*{A1}
\vor \(0 < a < b < \pi\)\\
\beh \(\int_a^b \cos(x) \cdot \ln\left(\sqrt{1 - \cos^2(x)}\right) = \left[\sin(x) \cdot \left(\ln(\sin(x)) - 1\right)\right]_a^b\)\\
\bew Es gilt: 
\begin{flalign*}
    & \int_a^b \cos(x) \cdot \ln\left(\sqrt{1 - \cos^2(x)}\right) & \text{| Logarithmusgesetze} \\
    & = \int_a^b \cos(x) \cdot \frac{1}{2}\ln\left(1 - \cos^2(x)\right) & \text{| \(\sin^2(x) + \cos^2(x) = 1\) (MatheB)} \\
    & = \int_a^b \cos(x) \cdot \frac{1}{2}\ln\left(1 - (1 - \sin^2(x))\right) & \text{| Klammern auflösen} \\
    & = \int_a^b \cos(x) \cdot \frac{1}{2}\ln\left(\sin^2(x)\right) & \text{| Logarithmusgesetze} \\
    & = \int_a^b \cos(x) \cdot \frac{1}{2} \cdot 2\ln\left(\sin(x)\right) & \\
    & = \int_a^b \cos(x) \cdot \ln\left(\sin(x)\right) & \\
    & = \int_a^b \ln\left(\sin(x)\right) \cdot \cos(x) & \text{| Partielle Integration} \\
    & = \left[\ln(\sin(x)) \cdot \sin(x)\right]_a^b - \int_a^b \left(\left(\frac{d}{dx}\ln(\sin(x))\right) \cdot \sin(x)\right) & \text{| Kettenregel} \\
    & = \left[\ln(\sin(x)) \cdot \sin(x)\right]_a^b - \int_a^b \left(\left(\frac{1}{\sin(x)} \cdot \frac{d}{dx}\sin(x)\right) \cdot \sin(x)\right) & \\
    & = \left[\ln(\sin(x)) \cdot \sin(x)\right]_a^b - \int_a^b \left(\left(\frac{1}{\sin(x)} \cdot \cos(x)\right) \cdot \sin(x)\right) & \\
    & = \left[\ln(\sin(x)) \cdot \sin(x)\right]_a^b - \int_a^b\cos(x) & \\
    & = \left[\ln(\sin(x)) \cdot \sin(x)\right]_a^b - \left[\sin(x)\right]_a^b & \text{| Reinziehen} \\
    & = \left[\ln(\sin(x)) \cdot \sin(x) - \sin(x)\right]_a^b & \text{| Ausklammern} \\
    & = \left[\sin(x) \cdot \left(\ln(\sin(x)) - 1\right)\right]_a^b
\end{flalign*}
\gedanke was zu zeigen war. \qed
\subsection*{A2}
\vor \\ 
\beh \\
\bew \\ \pagebreak
\subsection*{A3}
\vor \(\Omega := [0,1], \gap f: \Omega \rightarrow \R, x\mapsto \sqrt{x}, \gap\) \\
\beh \(f\) ist gleichmäßig stetig aber nicht Lipschitz-stetig. Also sind nicht alle gleichmäßig stetigen Funktionen Lipschitz-stetig.  \\
\bew \(f\) ist gleichmäßig stetig, da \(f\) stetig (MatheB) auf eine kompakte Menge ist, denn \(\Omega\) ist beschränkt und abgeschlossen.\\
Wir werden nun zeigen, dass \(f\) jedoch nicht Lipschitz-stetig ist, also \\
\(\forall L > 0: \exists x,y \in \Omega: |\sqrt{x} - \sqrt{y}| > L \cdot |x - y|\). \\
Sei \(L > 0\) beliebig und \(y := 0, x \ne 0\). Dann gilt: 
\begin{flalign*}
    & |\sqrt{x}| > L \cdot |x| & \text{| \(x > 0\)} \\
    \Longleftrightarrow & \sqrt{x} > L \cdot x & \text{| \(\cdot x^{-1}, \gap \cdot^{-1}\)} \\
    \Longleftrightarrow & \frac{x}{\sqrt{x}} < L^{-1} & \text{| \(\sqrt{x} = x^\frac{1}{2}\), Potenzgesetze} \\
    \Longleftrightarrow & \sqrt{x} < L^{-1} & \text{| \(\cdot^2, x > 0\)} \\
    \Longleftrightarrow & x < L^{-2}
\end{flalign*}
Somit existiert für jedes \(L > 0\) mindestens ein \(x, y\), sodass das Lipschitz-Kriterium bricht, also ist \(f\) nicht Lipschitz-stetig, \gedanke was zu zeigen war. \qed
\end{document}