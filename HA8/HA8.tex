\documentclass[12pt, a4paper]{article}

\usepackage[ngerman]{babel} 
\usepackage[T1]{fontenc}
\usepackage{amsfonts} 
\usepackage{setspace}
\usepackage{amsmath}
\usepackage{amssymb}
\usepackage{titling}
\usepackage{hyperref}


\newcommand*{\qed}{\null\nobreak\hfill\ensuremath{\square}}
\newcommand*{\puffer}{\text{ }\text{ }\text{ }\text{ }}
\newcommand*{\gedanke}{\textbf{-- }}
\newcommand*{\gap}{\text{ }}
\newcommand*{\setDef}{\gap|\gap}
\newcommand*{\vor}{\textbf{Vor.:} \gap}
\newcommand*{\beh}{\textbf{Beh.:} \gap}
\newcommand*{\bew}{\textbf{Bew.:} \gap}
% Hab länger gebraucht um zu realisieren, dass das ne gute Idee wäre
\newcommand*{\R}{\mathbb R}
\newcommand*{\grad}{\text{grad}}

\newenvironment{noalign*}
 {\setlength{\abovedisplayskip}{0pt}\setlength{\belowdisplayskip}{0pt}%
  \csname flalign*\endcsname}
 {\csname endflalign*\endcsname\ignorespacesafterend}


\pagestyle{plain}
\allowdisplaybreaks

\setlength{\droptitle}{-14em}
\setlength{\jot}{12pt}
\setlength{\hoffset}{-0.8in}

\title{Mathematik für die Informatik C\\Hausaufgabenserie 8}
\author{Henri Heyden, Nike Pulow \\ \small stu240825, stu239549}
\date{}


\begin{document}
\maketitle

\doublespacing
\subsubsection*{A1}
\subsubsection*{A2}
\vor Sei \(\phi: \R^3 \rightarrow \R, v \mapsto \dots\),\footnote[1]{Der Rest interessiert uns nicht, warum sehen wir später.}\\ \singlespacing
\(f: \R^4 \rightarrow \R^5, \begin{bmatrix}
    x \\ y \\ z \\ t
\end{bmatrix} \mapsto \begin{bmatrix}
    \dots \\ e^{-||(x,y,t)||^2} \\ \dots \\ \dots \\ \dots
\end{bmatrix} + z\sum_{i=1}^{5}e_i + \phi(t,z,x)\sum_{i = 1, i \ne 2}^{5}e_i\) \gap \footnote[2]{Auch hier interessieren uns Teile nicht.} \\ \\
\beh Es gilt:\\
\puffer \(\grad_{f_2}(x,y,z,t) = \begin{bmatrix}
    -2x \cdot e^{-x^2-y^2-t^2} \\ -2y \cdot e^{-x^2-y^2-t^2} \\ 1 \\ -2t \cdot e^{-x^2-y^2-t^2}
\end{bmatrix}\) \\
\bew Zuerst, sehen wir: \(z\sum_{i=1}^{5}e_i = z\begin{bmatrix}
    1 \\ 1 \\ 1 \\ 1 \\ 1
\end{bmatrix}\) \\ und: \(\phi(t,z,x)\sum_{i = 1, i \ne 2}^{5}e_i = \phi(t,z,x)\begin{bmatrix}
    1 \\ 0 \\ 1 \\ 1 \\ 1
\end{bmatrix}\) \\ \doublespacing \pagebreak \\
Da wir die zweite Komponentenfunktion betrachten, gilt somit: \\
\(f_2(x,y,z,t) = e^{-||(x,y,t)||^2} + z + 0 = e^{-(x^2+y^2+t^2)} + z = e^{-x^2-y^2-t^2} + z\).\footnote[3]{Hier sieht man warum die Teile uns egal sind: Da wir nur die zweite Komponentenfunktion betrachten, kürzt sich der Rest weg.} \\
Nach den bekannten Ableitungsregeln ergeben sich folgende partielle Ableitungen:
\begin{noalign*}
    \delta_1f_2(x) & = -2x \cdot e^{-x^2-y^2-t^2} \\
    \delta_2f_2(y) & = -2y \cdot e^{-x^2-y^2-t^2} \\
    \delta_3f_2(z) & = 1 \\
    \delta_4f_2(t) & = -2t \cdot e^{-x^2-y^2-t^2}
\end{noalign*}
Die Behauptung folgt. \qed
\subsubsection*{A3}
\vor \(\text R: \R \rightarrow \R^{2 \times 2}, \alpha \mapsto \begin{bmatrix}
    \cos(\alpha) & -\sin(\alpha) \\
    \sin(\alpha) & \cos(\alpha)
\end{bmatrix}\) \\ \\
\beh \(\forall \alpha, \beta \in \R: \text R(\alpha) \cdot \text R(\beta) = \text R(\alpha + \beta)\) \\
\bew Es gilt:
\begin{noalign*}
    & \text R(\alpha) \cdot \text R(\beta) = \begin{bmatrix}
        \cos(\alpha) & -\sin(\alpha) \\
        \sin(\alpha) & \cos(\alpha)
    \end{bmatrix} \cdot \begin{bmatrix}
        \cos(\beta) & -\sin(\beta) \\
        \sin(\beta) & \cos(\beta)
    \end{bmatrix} & \text{| Matrix Multiplikation} \\
    = & \begin{bmatrix}
        \cos(\alpha) \cdot \cos(\beta) - \sin(\alpha) \cdot \sin(\beta) & -\cos(\alpha) \cdot \sin(\beta) - \sin(\alpha) \cdot \cos(\beta) \\
        \sin(\alpha) \cdot \cos(\beta) + \cos(\alpha) \cdot \sin(\beta) & -\sin(\alpha) \cdot \sin(\beta) + \cos(\alpha) \cdot \cos(\beta)
    \end{bmatrix} & \text{| Ausklammern, Umstellen} \\
    = & \begin{bmatrix}
        \cos(\alpha) \cdot \cos(\beta) - \sin(\alpha) \cdot \sin(\beta) & -(\cos(\alpha) \cdot \sin(\beta) + \sin(\alpha) \cdot \cos(\beta)) \\
        \sin(\alpha) \cdot \cos(\beta) + \cos(\alpha) \cdot \sin(\beta) & \cos(\alpha) \cdot \cos(\beta) -\sin(\alpha) \cdot \sin(\beta)
    \end{bmatrix} & \text{| Additionstheoreme} \\
    = & \begin{bmatrix}
        \cos(\alpha + \beta) & -\sin(\alpha + \beta) \\
        \sin(\alpha + \beta) & \cos(\alpha + \beta)
    \end{bmatrix} = \text R(\alpha + \beta)
\end{noalign*} \\
\gedanke was zu zeigen war. \qed
\end{document}