\documentclass[12pt, a4paper]{article}

\usepackage[ngerman]{babel} 
\usepackage[T1]{fontenc}
\usepackage{amsfonts} 
\usepackage{setspace}
\usepackage{amsmath}
\usepackage{amssymb}
\usepackage{titling}
\usepackage{hyperref}


\newcommand*{\qed}{\null\nobreak\hfill\ensuremath{\square}}
\newcommand*{\puffer}{\text{ }\text{ }\text{ }\text{ }}
\newcommand*{\gedanke}{\textbf{-- }}
\newcommand*{\gap}{\text{ }}
\newcommand*{\setDef}{\gap|\gap}
\newcommand*{\vor}{\textbf{Vor.:} \gap}
\newcommand*{\beh}{\textbf{Beh.:} \gap}
\newcommand*{\bew}{\textbf{Bew.:} \gap}
% Hab länger gebraucht um zu realisieren, dass das ne gute Idee wäre
\newcommand*{\R}{\mathbb R}

\newenvironment{noalign*}
 {\setlength{\abovedisplayskip}{0pt}\setlength{\belowdisplayskip}{0pt}%
  \csname flalign*\endcsname}
 {\csname endflalign*\endcsname\ignorespacesafterend}


\pagestyle{plain}
\allowdisplaybreaks

\setlength{\droptitle}{-14em}
\setlength{\jot}{12pt}

\title{Mathematik für die Informatik C\\Hausaufgabenserie 7}
\author{Henri Heyden, Nike Pulow \\ \small stu240825, stu239549}
\date{}


\begin{document}
\maketitle

\doublespacing
\subsubsection*{A1}
\vor \(f : \R \rightarrow \R^2, (x,y) \mapsto
\begin{cases}
    xy\cdot \frac{x^2-y^2}{x^2+y^2} & (x,y) \ne (0,0)\\
    0 & \text{sonst}
\end{cases}\) \\
\beh \(f\) ist partiell differenzierbar in \(\R^2\) mit:\\
\[\delta_1f(x,y) = \begin{cases}
    \frac{y \cdot (x^4+4x^2y^2-y^4)}{(x^2+y^2)^2}& (x,y) \ne (0,0) \\
    0 & \text{sonst}
\end{cases}\] und:\\
\[\delta_2f(x,y) = \begin{cases}
    \frac{x \cdot (x^4-4x^2y^2-y^4)}{(x^2+y^2)^2}& (x,y) \ne (0,0) \\
    0 & \text{sonst}
\end{cases}\]\\ 
\bew Beachte für \(v \in \R^2 \setminus \{(0,0)\}\):\\
Es gilt: \(f = \pi_1\pi_2 \cdot \frac{\pi_1^2 - \pi_2^2}{\pi_1^2 + \pi_2^2}\) Somit ist \(f\) nach dem Kombinationssatz partiell differenzierbar auf \(\R^2 \setminus \{(0,0)\}\), da diese Menge offen ist. \\
Dann folgt aus der Quotientenregel und einmaligem Ausklammern die Behauptung für Differenzialwerte von \(f(x,y)\) mit \((x,y) \in \R^2 \setminus \{(0,0)\}\). \pagebreak \\
Nun für \((x,y) = (0,0)\): \\
\textbf{Fall 1:} \(x \rightarrow 0, y = 0\)
\begin{noalign*}
      & \lim_{x \rightarrow 0} \left(\frac{f(x,0) - (f(0,0) + 0)}{x - 0}\right) & \text{| \(f(0,0) = 0\)} \\
    = & \lim_{x \rightarrow 0} \left(\frac{f(x,0)}{x}\right) & \text{| Einsetzen} \\
    = & \lim_{x \rightarrow 0} \left(\frac{x \cdot 0 \cdot \frac{x^2-0}{x^2+0}}{x}\right) & \text{| Vereinfachen} \\
    = & \lim_{x \rightarrow 0} 0 \cdot 1 \\
    = & 0
\end{noalign*}
\textbf{Fall 2:} \(y \rightarrow 0, x = 0\)
\begin{noalign*}
      & \lim_{y \rightarrow 0} \left(\frac{f(0,y) - (f(0,0) + 0)}{|0 - y|}\right) & \text{| \(f(0,0) = 0\)} \\
    = & \lim_{y \rightarrow 0} \left(\frac{f(0,y)}{y}\right) & \text{| Einsetzen} \\
    = & \lim_{y \rightarrow 0} \left(\frac{y \cdot 0 \cdot \frac{-y^2}{y^2}}{y}\right) & \text{| Vereinfachen} \\
    = & \lim_{y \rightarrow 0} 0 \cdot -1 \\
    = & 0
\end{noalign*}
Somit ist die Behauptung für Differenzialwerte von \(f(x,y)\) mit\\
\((x,y) \in \{(0,0)\}\) auch gezeigt. Die Behauptung ist somit gezeigt. \qed
\subsubsection*{A2}
\vor \(V,W\) sind normierte Räume, \(\phi \in \text L(V,W)\), \(b \in W\),\\
\puffer\puffer\gap\(f : V \rightarrow W, v \mapsto \phi(v) + b\) \\
\beh \(f\) ist differenzierbar für alle \(v \in V\) mit \(D_{V,W}f(v) = \phi\) \\
\bew Da \(f = \phi + b\) gilt (hier \(b\) als Konstante Funktion), und \(\phi, b\) differenzierbar sind, ist \(f\) differenzierbar.\\
Es ergibt sich folgender Grenzwert:
\begin{noalign*}
      & \lim_{\tilde{v} \rightarrow v} \left(\frac{f(\tilde{v}) - (f(v) + \phi(\tilde{v} - v))}{||\tilde{v} - v||_V}\right) & \text{| Einsetzen} \\
    = & \lim_{\tilde{v} \rightarrow v} \left(\frac{\phi(\tilde{v}) + b - (\phi(v) + b + \phi(\tilde{v} - v))}{||\tilde{v} - v||_V}\right) & \text{| Vereinfachen und ausklammern} \\
    = & \lim_{\tilde{v} \rightarrow v} \left(\frac{\phi(\tilde{v}) - \phi(v) - \phi(\tilde{v} - v)}{||\tilde{v} - v||_V}\right) & \text{| Additivität, Homogenität, ausklammern} \\
    = & \lim_{\tilde{v} \rightarrow v} \left(\frac{\phi(\tilde{v}) - \phi(v) - \phi(\tilde{v}) + \phi(v)}{||\tilde{v} - v||_V}\right) & \text{| Vereinfachen} \\
    = & \lim_{\tilde{v} \rightarrow v} \left(\frac{0}{||\tilde{v} - v||_V}\right) \\
    = & 0
\end{noalign*}
Also ist \(\phi\) Differenzialquotient von \(f\). \qed
\subsubsection*{A3}
\vor \\
\beh \\
\bew \\
\end{document}